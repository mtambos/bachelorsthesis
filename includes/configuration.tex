% header.tex

\usepackage[a4paper,left=3.5cm,right=2.5cm,bottom=3.5cm,top=3cm]{geometry}
% \usepackage[german,english]{babel}
\usepackage[pdftex]{color}
\usepackage{amsmath,mathtools,amssymb}
%\usepackage{subfigure}

% Theorem-Umgebungen
\usepackage[amsmath,thmmarks]{ntheorem}

% Korrekte Darstellung der Umlaute
% \usepackage[utf8]{inputenc}
% \usepackage[T1]{fontenc}

% Algorithmen
\usepackage[plain,chapter]{algorithm}
\usepackage{algorithmic}
\usepackage{enumerate}

% refstyle
\usepackage{refstyle}
% \newref{chapter}{name=chapter~,Name=Chapter~,names=chapters~,Names=Chapters~}
\newref{sec}{name=section~,Name=Section~,names=sections~,Names=Sections~}
% \newref{subsection}{name=subsection~,Name=Subsection~,names=subsections~,Names=Subsections~}
\newref{cor}{name=corollary~,Name=Corollary~,names=corollaries~,Names=Corollaries~}
\newref{def}{name=definition~,Name=Definition~,names=definitions~,Names=Definitions~}
\newref{sub}{name=subsection~,Name=Subsection~,names=subsections~,Names=Subsections~}
\newref{src}{name=source code~,Name=Source code~,names=source codes~,Names=Source codes~}

% Bibtex deutsch
\usepackage{bibgerm}

% URLs
\usepackage{url}

% Caption Packet
\usepackage[margin=0pt,font=small,labelfont=bf]{caption}
% Gliederung einstellen
%\setcounter{secnumdepth}{5}
%\setcounter{tocdepth}{5}

\usepackage[explicit]{titlesec}
\usepackage{lmodern}
\usepackage{lipsum}
\usepackage{listings}


% Abstand swischen zwei Absätzen
%\parskip=0.5em

\newlength\chapnumb
\setlength\chapnumb{2.5cm}

\titleformat{\chapter}[block]
{\normalfont\sffamily}{}{0pt}
{\parbox[b]{\chapnumb}{%
   \fontsize{80}{70}\selectfont\thechapter}%
  \parbox[b]{\dimexpr\textwidth-\chapnumb\relax}{%
    \raggedleft%
    \hfill{\LARGE#1}\\
    \rule{\dimexpr\textwidth-\chapnumb\relax}{0.4pt}}}
\titleformat{name=\chapter,numberless}[block]
{\normalfont\sffamily}{}{0pt}
{\parbox[b]{\dimexpr\textwidth\relax}{%
    \raggedleft%
    \hfill{\LARGE#1}\\
    \rule{\dimexpr\textwidth\relax}{0.4pt}}}

% Theorem-Optionen %
\theoremseparator{.}
\theoremstyle{change}
\newtheorem{theorem}{Theorem}[section]
\newtheorem{lemma}[theorem]{Lemma}
\newtheorem{corollary}[theorem]{Corollary}
\newtheorem{proposition}[theorem]{Proposition}
% Ohne Numerierung
\theoremstyle{nonumberplain}
\renewtheorem{theorem*}{Theorem}
\renewtheorem{lemma*}{Lemma}
\renewtheorem{corollary*}{Corollary}
\renewtheorem{proposition*}{Proposition}
% Definitionen mit \upshape
\theorembodyfont{\upshape}
\theoremstyle{change}
\newtheorem{definition}[theorem]{Definition}
\theoremstyle{nonumberplain}
\renewtheorem{definition*}{Definition}
% Kursive Schrift
\theoremheaderfont{\itshape}
\newtheorem{notation}{Notation}
\newtheorem{convention}{Convention}
\newtheorem{designation}{Designation}
\theoremsymbol{\ensuremath{\Box}}
\newtheorem{proof}{Proof}
\theoremsymbol{}
\theoremstyle{change}
\theoremheaderfont{\bfseries}
\newtheorem{remark}[theorem]{Remark}
\newtheorem{observation}[theorem]{Observation}
\newtheorem{example}[theorem]{Example}
\newtheorem{problem}{Problem}
\theoremstyle{nonumberplain}
\renewtheorem{observation*}{Observation}
\renewtheorem{example*}{Example}
\renewtheorem{problem*}{Problem}

% Algorithmen anpassen %
\renewcommand{\algorithmicrequire}{\textit{Input:}}
\renewcommand{\algorithmicensure}{\textit{Output:}}
\floatname{algorithm}{Algorithm}
\renewcommand{\listalgorithmname}{Algorithm List}
\renewcommand{\algorithmiccomment}[1]{\color{grey}{// #1}}

% Zeilenabstand einstellen %
\renewcommand{\baselinestretch}{1.25}

% Floating-Umgebungen anpassen %
\renewcommand{\topfraction}{0.9}
\renewcommand{\bottomfraction}{0.8}

% Abkuerzungen richtig formatieren %
\usepackage{xspace}

% Leere Seite ohne Seitennummer, naechste Seite rechts
\newcommand{\blankpage}{
 \clearpage{\pagestyle{empty}\cleardoublepage}
}

% Keine einzelnen Zeilen beim Anfang eines Abschnitts (Schusterjungen)
\clubpenalty = 10000

% Keine einzelnen Zeilen am Ende eines Abschnitts (Hurenkinder)
\widowpenalty = 10000 
\displaywidowpenalty = 10000

% EOF
